\subsection{Speicherverwaltung}

\lstinputlisting{../if/ifmemoryserver.idl}

\subsubsection{Pager}

\begin{center}
\begin{tikzpicture}[block/.style={rectangle,very thick,draw=kit-blue100,top color=white,bottom color=kit-blue30,minimum size=10mm,align=center}]

	\node at (0,0) [block] (sigma0) {Sigma0};
	\node at (0,-2) [block] (minipager) {minipager};

	\node at (-2.5,-4) [block] (memoryserver) {memoryserver};
	\node at (0,-4) [block] (taskserver) {taskserver};
	\node at (2,-4) [block] (nameserver) {nameserver};
	\node at (4,-4) [block] (fileserver) {fileserver};

	\node at (-5,-6) [block] (shell1) {Shell 1};
	\node at (-2.5,-6) [block] (simplethread1) {simplethread1};
	\node at (0,-6) [block] (again) {...};

	\draw[very thick,<->] (sigma0.south) -- (minipager.north);

	\draw[very thick,<->] (minipager.south) -- (memoryserver.north);
	\draw[very thick,<->] (sigma0.west) -- (memoryserver.north);
	\draw[very thick,<->] (minipager.south) -- (taskserver.north);
	\draw[very thick,<->] (minipager.south) -- (nameserver.north);
	\draw[very thick,<->] (minipager.south) -- (fileserver.north);

	\draw[very thick,<->] (memoryserver.south) -- (shell1.north);
	\draw[very thick,<->] (memoryserver.south) -- (simplethread1.north);
	\draw[very thick,<->] (memoryserver.south) -- (again.north);
\end{tikzpicture}
\end{center}

\begin{itemize}
	\item Minipager: Nur zuständig für die Kern-Server
	
	\item Der Thread verwaltet seinen Adressraum selbst. Dies ist die bessere Idee, da dies den Pager sehr vereinfacht und die Möglichkeiten des Threads maximiert, z.B. könnten dynamisch Libraries nachgeladen werden, ohne dass der Pager dies wissen muss. Allerdings muss  der Thread selbst entscheiden, wo in seinem Adressraum das Mapping landen soll. Dies setzt voraus, dass der Thread ein lokales Heap-Management hat, welches selbst entscheiden kann, an welche freie Adresse etwas eingeblendet werden soll.
	
	\item free: Der Pager ruft ein rekursives Flush auf die Page auf, dadurch werden alle Mappings entfernt, und die Seite an Sigma0 zurückgegeben.
    \item Starten eines Tasks: Um einen Thread zu starten, wird zunächst
        die create\_task Methode des Task-Servers aufgerufen. Dieser
        kommuniziert mit dem Loader, um die Teile der ELF-File zu
        finden, die eingeblendet werden müssen. Für diese Teile der
        ELF-File werden beim Memory-Server file-mappings angelegt.
        Anschließend wird die Startup\_IPC durch den Memoryserver
        gesendet.
\end{itemize}

\subsubsection{Speicherlayout}
\begin{tabular}{|l|l|}
\hline
Adresse & Area \\
\hline
0xFFFFFFFF .. 0xC0000001 & Kernel \\
\hline
0xC0000000 .. 0xBF000001 & UTCB \\
\hline
0xBF000000 .. 0xB0000001 & KIP \\
\hline
0xB0000000 .. 0xA0000001 & empty \\
\hline
0xA0000000 ... & STACK \\
\hline
... 0x98000000 & HEAP \\
\hline
0x97FFFFFF .. 0x78000000 & APP (Text, BSS, Data) \\
\hline
\end{tabular}
\subsubsection{Lokale Heap-Verwaltung}

Vorhandene Heap-Bibliothek portieren, z.B. dietlibc oder dlmalloc. dietlibc malloc()/free() ist bereits portiert. dlmalloc wurde von Jens Kehne (Mitarbeiter von Herrn Dr. Stoess) empfohlen.


\subsection{Console Server}

\lstinputlisting{../if/ifconsoleserver.idl}

Funktionsumfang:
\begin{itemize}
	\item Virtuelle Terminals (aktuell 8)
	\item Buffering von In- und Output
	\item Kein dedizierter Bildschirmtreiber, sondern direkter Zugriff auf den Video-Speicher bei 0xb8000
	\item Kein dedizierter Tastaturtreiber, sondern Interrupt-Routine für INT1
\end{itemize}

Jedes virtuelle Terminal hat einen eigenen Bildschirm- und Tastaturpuffer, sowie eine aktive ThreadID, welche den Task identifiziert, welcher dem virtuellen Terminal zugeordnet ist. Ruft ein Thread put*() oder get*(), so beziehen sich diese nur auf das Terminal, welches dem Task, zu dem der Thread gehört, zugeordnet ist. Ist der Aufrufer dem Console Server unbekannt, endet die Anfrage ohne Ergebnis.

Startet ein Thread einen neuen Task, so ist der Thread dafür verantwortlich, die Kontrolle über das Terminal an den neuen Task zu übergeben (z.B. im Falle einer Shell).

Das Umschalten zwischen den Konsolen funktioniert via Ctrl+[1-9].
\subsection{Task Server}

\lstinputlisting{../if/iftaskserver.idl}


\subsubsection{Task-IDs und Thread-IDs}

Eine globale Thread-ID besteht aus der Task-ID (vordere x Bits) und einem Zähler (hintere x Bits). Der Zähler nummeriert die Threads innerhalb eines Tasks durch. Die ThreadId des Adressraumes des Tasks, die ThreadId des ersten Threads im Task und die TaskId sind identisch. Beispiels:

\begin{itemize}
	\item Der Zähler ist 8 Bit lang
	\item Die TaskId ist 10 Bit lang
	\item Globale ThreadId Nummer 0x00070 : Task 0x00070, Adressraum 0x00070, Thread 0x00070, Thread Nummer 0 innerhalb des Tasks
	\item Globale ThreadId Nummer 0x00077 : Task 0x00070, Adressraum 0x00070, Thread 0x00077, Thread Nummer 7 innerhalb des Tasks
\end{itemize}

Die von L4 definierte Local Thread Id ist etwas völlig anderes!
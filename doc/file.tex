\subsection{File server}

\lstinputlisting{../if/iffile.idl}

\lstinputlisting{../if/iffileserver.idl}

\begin{itemize}
	\item Bereitstellung der von GRUB geladenen Module als virtuelles Dateisystem.
	
	\item API: Datenübergabe über Puffer (sequence<char>)
\end{itemize}

\subsubsection{Namensauflösung der Pfade}

Nähere Beschreibung zur Funktionsweise des lookup siehe Kapitel "Naming".

\subsubsection{FileID holen}

Die Funktion get\_file\_id wertet den übergebenen Pfad aus und liefert bei Übereinstimmung
die entsprechende ID, die über einen zero-based Index repräsentiert wird.

Auszug des virtullen Dateisystems:
\begin{lstlisting}
Module: start afb000 size 1ff4a type: 1 cmdline: (cd)/sdios/nameserver
Module: start b1b000 size 216e2 type: 1 cmdline: (cd)/sdios/memoryserver
Module: start b3d000 size 27c22 type: 1 cmdline: (cd)/sdios/fileserver
Module: start baf000 size d1f1 type: 1 cmdline: (cd)/sdios/keyboarddriver
\end{lstlisting}

Anwendungsbeispiel:

Der Aufruf von get\_file\_id mit dem Parameter "memoryserver" liefert einen Rückgabewert von 1 zurück.

\subsubsection{Directory listings}

Code für die Auflistung des Inhalts von "/":

\begin{lstlisting}
path_t outbuf[];

int entries = get_dir_size("/");

for(int i = 0; i < entries; i++)
{
  get_dir_entry("/", i, outbuf);
  
  int type = get_file_type(outbuf);
  int size = get_file_size(outbuf);
  
  printf("Entry: %s Type: %i size: %i\n", outbuf, size, type);
}
\end{lstlisting}


\section{Design}

\begin{center}
\begin{tikzpicture}[block/.style={rectangle,very thick,draw=kit-blue100,top color=white,bottom color=kit-blue30,minimum size=10mm,align=center}]

	\node at (-5,0) (schicht1) {Schicht 1};
	\node at (-5,1.5) (schicht2) {Schicht 2};
	\node at (-5,3) (schicht3) {Schicht 3};

	\node at (-2,0) [block] (logger) {Logger};
	\node at (0,0) [block] (minipager) {Minipager};
	\node at (2,0) [block] (task) {Task};

	\draw (-4,0.7) -- (5,0.7) node () {};

	\node at (-2,1.5) [block] (nameserver) {Nameserver};
	\node at (0,1.5) [block] (console) {Console};
	\node at (2,1.5) [block] (screen) {Memory};
	\node at (4,1.5) [block] (file) {File};

	\draw (-4,2.2) -- (5,2.2) node () {};

	\node at (-2,3) [block] (shell1) {Shell 1};
	\node at (0,3) [block] (shell2) {Shell 2};
	\node at (2,3) [block] (shell3) {Task 1};
	\node at (4,3) [block] (task) {Task 2};
\end{tikzpicture}
\end{center}

\subsection{Schicht 1 (Roottask)}

Im Adressraum der Roottask laufen der Logger und der Minipager. Der Minipager wird später zum Pager für die notwendigsten System-Server und reicht physikalische Speicherseiten idempotent durch. Nach erfolgter Initialisierung der Startumgebung wird die Roottask selbst zum Taskserver.

\subsection{Schicht 2}

In dieser Schicht befinden sich die notwendigsten System-Server und Treiber.

\subsection{Schicht 3}

In dieser Schicht befinden sich die User-Tasks, welche über den Taskserver gestartet werden und Ein-/Ausgaben über den Consoleserver tätigen.
